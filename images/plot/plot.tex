\documentclass[border=3pt]{standalone}

% Circuits
\usepackage[european,straightvoltages, RPvoltages, americanresistor, americaninductors]{circuitikz}
\tikzset{every picture/.style={line width=0.2mm}}

% Notation
\usepackage{amsmath}

% Tikz Library
\usetikzlibrary{calc}

% Bipoles Specifications
\ctikzset{bipoles/thickness=1.2, label distance=1mm, voltage shift = 1}

% Arrows Above Compenents
% Source: https://tex.stackexchange.com/questions/574576/circuitikz-straight-voltage-arrows-with-fixed-length
\newcommand{\fixedvlen}[3][0.4cm]{% [semilength]{node}{label}
    % get the center of the standard arrow
    \coordinate (#2-Vcenter) at ($(#2-Vfrom)!0.5!(#2-Vto)$);
    % draw an arrow of a fixed size around that center and on the same line
    \draw[-{Triangle[round,open]}] ($(#2-Vcenter)!#1!(#2-Vfrom)$) -- ($(#2-Vcenter)!#1!(#2-Vto)$);
    % position the label as it where if standard voltages were used
    \node[ anchor=\ctikzgetanchor{#2}{Vlab}] at (#2-Vlab) {#3};
}
\newcommand{\fixedvlendashed}[3][0.75cm]{% [semilength]{node}{label}
    % get the center of the standard arrow
    \coordinate (#2-Vcenter) at ($(#2-Vfrom)!0.5!(#2-Vto)$);
    % draw an arrow of a fixed size around that center and on the same line
    \draw[dashed,-{Triangle[round,open]}] ($(#2-Vcenter)!#1!(#2-Vfrom)$) -- ($(#2-Vcenter)!#1!(#2-Vto)$);
    % position the label as it where if standard voltages were used
    \node[ anchor=\ctikzgetanchor{#2}{Vlab}] at (#2-Vlab) {#3};
}

\begin{document}
	
	\begin{circuitikz}
		
		%Circuit
		\node[op amp] at (0,0) (opamp) {};
		\node[ground] at (-3.99,-2.4) (ground1) {};
		\node[ground] at (-1.65,-2.4) (ground2) {};
		\node[ground] at (-6,-2.4) (ground3) {};
		\draw (opamp.-) to[R, l_=$R_3$, name=R3] ++(-2.8,0) to[R, l=$R_2$, name=R2] ++(0,-2.6) -- (ground1);
		\draw (opamp.+) -- ++(-0.46,0) -- ++(0,-1.5) --(ground2);
		\draw (ground3) to[mic, name=M] ++(0,3) to[R, l=$R_1$, name=R1] ++(0,2);
		\draw [-latex] (R1) -- ++(0,1.6) node[above] {$V_{CC}$};
		\draw ($(opamp.-)+(-2.79,0)$) to[short, *-] ++(-0.5,0) to[C, l=$C$, name=C] ++(-1,0) to[short,-*] ++(-0.52,0);
		\draw ($(opamp.-)+(-0.45,0)$) to[short, *-] ++(0,2.3) to[R, l_=$R_4$, name=R4] ++(3.3,0) to[short,-*] ++(0,-2.8);
		\draw (opamp.out) to[short,-*] ++(1.5,0) node[shift={(0.4,0)}] {$v_\text{O}$};
		\draw[-latex] (opamp.up) -- ++(0,0.6) node[above] {$+V_{S}$};
		\draw (opamp.down) -- ++(0,-1.2) to[short,-*] ++(-1.56,0);
		
		
		%Nodes
		\node[shift={(0,0.3)}] at (opamp.-) {2};
		\node[shift={(0,-0.4)}] at (opamp.+) {3};
		\node[shift={(0.2,0.2)}] at (opamp.up) {7};
		\node[shift={(0.2,-0.2)}] at (opamp.down) {4};

	\end{circuitikz}
	
\end{document}